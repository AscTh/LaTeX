\documentclass[12pt]{article}
\RequirePackage[l2tabu, orthodox]{nag}
\usepackage[utf8]{inputenc}
\usepackage[russian]{babel}
\usepackage[top=2cm, bottom=1.5cm, left=2.5cm, right=2.5cm]{geometry}
%\usepackage{amsmath}
\usepackage{amssymb}
\usepackage{amsthm}
\usepackage{framed}
\usepackage{mathtools}
\usepackage{microtype}
%\usepackage{oubraces}
\usepackage{tikz}
%\usepackage{stmaryrd}
\numberwithin{equation}{section}
\newtheorem{theorem}{Теорема}[section]
\newtheorem{lemma}{Лемма}[section]
\newtheorem{remark}{Следствие}[section]
\usepackage{titlesec}
\usepackage{titletoc}
\usepackage{hyperref}

\hypersetup{
    colorlinks=true,
    linkcolor=black,
    filecolor=magenta,      
    urlcolor=black,
    pdftitle={Sharelatex Example},
    bookmarks=true,
}
%\setcounter{secnumdepth}{1} %% Не нумеровать вопросы, они ведь и так!

\titleformat{\section}
  {\Large\bfseries} % format
  {}                % label
  {0pt}             % sep
  {\Large}           % before-code


\title{Экзамен по математическому анализу.\\Второй семестр}
\date{}


\begin{document}

\maketitle
\tableofcontents
\thispagestyle{empty}

\newpage

\section{Вопрос №1}  %%%%%%%%%%%%%%%%% ВОПРОС 1
\begin{framed}
Теорема о разложении правильной рациональной дроби на простейшие и ее доказательство.\\
\end{framed}
\begin{theorem}[Основная теорема алгебры] Любой многочлен $Q_n(x)$ степеней и $n$ с действительными коэффициентами представляется в виде
\begin{equation}
Q_n(x) = a_0(x-c_1)^{k_1} \cdot \ldots \cdot (x-c_r)^{k_r}(x^2 + p_1x + q_1)^{s_1} \cdot \ldots \cdot (x^2 + p_tx + q_t)^{s_t}, \label{eq:1}
\end{equation}

где $c_i , \ldots , c_r$ — действительные корни многочлена $Q_n(x)$, квадратные трехчлены $x^2 + p_i x + q_i$ не имеют действительных корней, $k_1 + \dots + k_r + 2(s_1 + \dots + s_t) = n$. 
\end{theorem}


\begin{theorem}
Всякая правильная рациональная дробь представляется в виде
\[ \frac{P_m(x)}{Q_n(x)} = \frac{A_{1,1}}{(x-c_1)^{k_1}} + \dots + \frac{A_{1,k_1}}{x-c_1} + \dots + \left[ \text{аналогично для }c_2, \ldots, c_r\right] + \]
\[ + \frac{M_{1,1}x + N_{1,1}}{(x^2 + p_1x + q_1)^{s_1}} + \dots + \frac{M_{1,s_1}x + N_{1,s_1}}{x^2 + p_1x + q_1} + \dots + \left[ \frac{\text{аналогично для}}{(p_2,q_2), \ldots, (p_t,q_t)}\right]. \]\\

Где в знаменателях стоят сомножетели разложения $\eqref{eq:1}, A_{i,j} , M_{i,j} , N_{i,j}$ — некоторые
числа, зависящие от $P_m(x)$ и $Q_n(x)$.
\end{theorem}

Прежде чем доказать теорему о разложении правильной рациональной дроби сформулируем и докажем две леммы.
\begin{lemma} \label{lemm:1:1}
Если число $a \in \mathbb{R}$  является действительным корнем кратности $k$ многочлена $Q_n(x) = (x-a)^k Q_{n-k}(x)$, то несократимую правильную рациоальную дробь $\dfrac{P_m(x)}{Q_n(x)}$ можно представить в виде

\[\frac{P_m(x)}{Q_n(x)} = \frac{A}{(x-a)^k} + \frac{P_l(x)}{(x-a)^{k-1}Q_{n-k}(x)}, \]

где $A \neq 0$ и $P_l(x)$ ---  многочлен cтепени $l < n - 1$,  т.e. последнее слагаемое является правильной рациональной дробью.
\end{lemma}
\begin{proof}
Добавим к дроби слагаемые $\dfrac{A}{(x-a)^k}~(A \in \mathbb{R})$ с разными знаками и преобразуем выражение:
\[ \frac{P_m(x)}{Q_n(x)} =\frac{P_m(x)}{(x-a)^k Q_{n-k}(x)} + \frac{A}{(x-a)^k} - \frac{A}{(x-a)^k} =  \frac{A}{(x-a)^k} + \frac{P_m(x) - AQ_{n-k}(x)}{(x-a)^k Q_{n-k}(x)}.\]

Второе слагаемое является правильной рациональной дробью, т.к $m < n$ и $n -k < n$. Так как $a$ — корень кратности $k$, то $Q_{n-k}(a) \neq 0$. Так как $\dfrac{P_m(x)}{Q_n(x)}$ --- несократимая дробь, то $P_m(a) \neq 0$. Положим $A = \dfrac{P_m(a)}{Q_{n-k}(a)} \neq 0$. Тогда $P_m(a) - AQ_{n-k}(a) = 0$, а значит, по теореме Безу многочлен $P_m(x) - A Q_{n-k}(x)$ делится на $x-a$. Сократим второе слагаемое на $x-a$:
\begin{equation}
\frac{P_m(x) - AQ_{n-k}(x)}{(x-a)^kQ_{n-k}(x)} = \frac{P_l(x)}{(x-a)^{k-1} Q_{n-k}(x)}. \label{eq:2}
\end{equation}

Так как эта дробь получена сокращением правильной рациональной дроби, она является правильной (но, возможно, сократимой). 
\end{proof}

\begin{lemma} \label{lemm:1:2}
Если комплексное число $z = \alpha + \beta i~(\beta \neq 0)$ является корнем кратности $s \in \mathbb{N}~(2 \leqslant 2s \leqslant n) $ многочлена $Q_n(x)$ действительными коэффициентами, то несократимую правильную рациональную дробь $\dfrac{P_m(x)}{Q_n(x)}$ можно представить в виде
\begin{equation}
\frac{P_m(x)}{Q_n(x)} = \frac{Mx + N}{(x^2 + px + q)^s} + \frac{P_l(x)}{(x^2 + px + q)^{s-1} Q_{n-2s}(x)}, \label{eq:3}
\end{equation}

где $x^2 + px + q = (x-z)(x-\bar{z}), Q_{n-2s}(x)$ --- такой многочлен степени $n- 2s$, что
\[ Q_n(x) = (x^2 + px + q)^s Q_{n-2s}(x),~~~~~Q_{n-2s}(z) \neq 0, \]

числа $M, N$ одновременно не обращаются в нуль, а последнее слагаемое в $\eqref{eq:3}$ является правильной рациональной дробью: $l < n - 2$.
\end{lemma}
\begin{proof}
Добавим к дроби слагаемые $\dfrac{(Mx + N)}{(x^2 + px + q)^s}~(M,N \in \mathbb{R})$ с разными знаками и преобразуем выражение:
\begin{equation}
\frac{P_m(x)}{Q_n(x)} =\frac{Mx + N}{(x^2 + px + q)^s} + \frac{P_m(x) - (Mx+n)Q_{n-2s}(x)}{(x^2 + px + q)^s Q_{n-2s}(x)}. \label{eq:4}
\end{equation}

В знаменателе второго слагаемого в правой части равенства $\eqref{eq:4}$ стоит многочлен степени $n$, а степень многочлена в числителе этого слагаемого меньше $n$, поскольку и $m < n$, и $n - 2s + 1 < n$. Следовательно, это слагаемое является правильной рациональной дробью.\\

Выберем действительные числа $M$ и $N$ так, чтобы многочлен $P_m(x) - (Mx + N) Q_{n-2s}(x)$ делился на $x^2 + px + q$. Это эквивалентно тому, что
\begin{equation}
P_m(z) - (Mz + N) Q_{n-2s}(z) = 0, ~~~~~ z = \alpha + \beta i. \label{eq:5}
\end{equation}

Тогда и сопряженное $z$ комплексное число $\bar{z} = \alpha - \beta i$ также будет корнем этого уравнения. Поэтому многочлен $P_m(x) - (Mx + N) Q_{n-2s}(x)$ делится на $x^2 + px + q = (x-z)(x-\bar{z})$. Из $\eqref{eq:5}$ следует, что 
\[ M(\alpha + \beta i) + N = \frac{P_m(z)}{Q_{n-2s}(z)} = K + Li,   \]

где $K$ и $L$ --- некоторые действительные числа. Приравнивая в этом равенстве действительные и мнимые части, получаем $M \alpha + N = K$ и $M \beta = L$. Отсюда $M = \dfrac{L}{\beta} $ и $N = K - \dfrac{\alpha L}{\beta}$. Числа $M$ и $N$ одновременно не обращаются в нуль, так как в противном случае $P_m(z) = 0$ и дробь $\dfrac{P_m(x)}{Q_n(x)}$ сократима, а это противоречит условию леммы. При таком выборе $M$ и $N$ второе слагаемое в правой части $\eqref{eq:4}$ можно сократить на $x^2 + px + q$, записав его в виде
\begin{equation}
\frac{P_m(x) - (Mx + N) Q_{n-2s} (x)}{(x^2 + px + q)^s Q_{n-2s}(x)} = \frac{P_l(x)}{(x^2 + px + q)^{s-1} Q_{n-2s}(x)}, \label{eq:6}
\end{equation}

где $P_l(x) = \dfrac{P_m(x) - (Mx + N) Q_{n-2s}(x)}{x^2 + px + q}$. Рациональная дробь в правой части $\eqref{eq:6}$ получена сокращением правильной рациональной дроби с действительными коэффициентами на множитель $x^2 + px + q$, где $p$ и $q$ --- действительные числа, и поэтому является правильной рациональной дробью с действительными коэффициентами.
\end{proof}

\begin{proof}[Доказательство теоремы]
К правильной рациональной дроби в правой части $\eqref{eq:2}$ при $k > 1$ можно вновь применить лемму \ref{lemm:1:1} и в итоге получить
\begin{equation}
\frac{P_m(x)}{Q_n(x)} = \frac{A}{(x-a)^k} + \frac{A_1}{(x-a)^{k-1}} + \ldots + \frac{A_{k-1}}{x-a} + \frac{P_{l_1}(x)}{Q_{n-k}(x)}.\label{eq:7}
\end{equation}

Здесь $\dfrac{P_{l_1}(x)}{Q_{n-k}(x)}$ --- несократимая правильная рациональная дробь, так как в ином случае правая часть в $\eqref{eq:7}$ после ее приведения к общему знаменателю является сократимой дробью, а это противоречит условию леммы. Если многочлен $Q_{n-k}(x)$ имеет другие действительные корни, то к последней дроби в правой части $\eqref{eq:7}$ также применима лемма \ref{lemm:1:1}. И т.д.\\

Если многочлен $Q_n(x)$  не имеет действительных корней, а имеет комплексные корни, то применима лемма \ref{lemm:1:2}. Ко второму слагаемому в правой части $\eqref{eq:3}$ при $s > 1$ можно вновь применить лемму \ref{lemm:1:2} и получить
\[ \frac{P_m(x)}{Q_n(x)} = \frac{Mx + N}{(x^2 + px + q)^s} + \frac{M_1x + N_1}{(x^2 + px + q)^{s-1}} + \ldots + \frac{M_{s-1}x + N_{s-1}}{x^2 + px + q} + \frac{P_s(x)}{Q_{n-2s}(x)},\]

где $\dfrac{P_s(x)}{Q_{n-2s} (x)}$ ---  несократимая правильная рациональная дробь. Многочлен $Q_{n-2s}(x)$ имеет другие комплексные корни, а значит, к этой дроби также применима лемма \ref{lemm:1:2}. И т.д.
\end{proof}

\section{Вопрос №2} %%%%%%%%%%%%%%%%% ВОПРОС 2
\begin{framed}Интегрирование простейших дробей и рациональных функций.
\end{framed}
\textbf{Рациональными} называют функции вида
\[f(x) = \frac{P_m(x)}{Q_n(x)},~~~~
\begin{aligned}
P_m(x) &= b_0x^m + b_1 x^{m-1} + \ldots + b_{m-1}x +b_m,\\
Q_n(x) &= a_0 x^n + a_1 x^{n-1} + \ldots + a_{n-1}x + a_n.
 \end{aligned}\]

\textbf{Простейшие рациональные дроби:}
\[ 1)~\frac{A}{x-a},~~~2)~\frac{B}{(x-a)^k},~~~3)~\frac{Mx + N}{x^2 + px + q}~~~4)~\frac{Mx+N}{(x^2 + px + q)^k},~~~k>1\]

Схема интегрирования рациональных функций: сначала функцию представляют в виде суммы простейших дробей, а затем интегрируют каждое слагаемое полученного разложения.

Если $n > m \geqslant 0$,  рациональную дробь называют \textbf{правильной}, в противном случае --- \textbf{неправильной}. Используя правило деления многочленов, неправильноую рациональную дробь можно предстваить в виде суммы многочлена $P_{m-n}$ степени  $m-n$ и некоторой правильной дроби, т.е.
\[ \frac{P_m(x)}{Q_n(x)} = P_{m-n}(x) + \frac{P_l(x)}{Q_n(x)},\]

где многочлен $P_l(x)$ имеет степень $l < n$.

\begin{theorem}[Теорема Безу]
Если $a$ --- корень многочлена $P(x)$, то многочлен $P(x)$ делится на $x-a$ без остатка.
\end{theorem}
 
Натуральное число $k$ называют \textbf{кратностью} корня $a$ многочлена $P(x)$, если существует такой многочлен $Q(x)$, что
\[ P(x) = (x-a)^k Q(x),~~~~Q(a) \neq 0.\]

Теорема Безу верна и в случае действительного, и в случае комплексного корня. Пусть многочлен $Q_n(x)$ степени $n$ с действительными коэффициентами имеет комплексный корень $z = \alpha + \beta i~(\alpha, \beta \in \mathbb{R},~\beta \neq 0$, $i$ --- \textbf{мнимая единица}, $i^2 = -1$) кратности $k \in \mathbb{N}$. Но тогда сопряженное с этим корнем комплексное число $\bar{z} = \alpha - \beta i$ является для данного многочлена корнем той же кратности, т.е. многочлен $Q_n(x)$ делится на многочлен
\[ ((x-z)(x-\bar{z}))^k = (x^2 + px + q)^k,\]

где $p = -2a$ и $q = \alpha^2 + \beta^2$ --- действительные числа.\\

\textbf{Алгоритм интегрирования рациональных функций:}

\begin{enumerate}
\item [0.] В случае, если необходимо проинтегрировать недробное рациональное выражение, перейти к шагу 4
\item Выделить правильную дробь (если необходимо)
\item Найти вид разложения на простейшие дроби
\item Найти коэффициенты разложения
\item Проинтегрировать каждое слагаемое
    \begin{enumerate}
    \item [4.1] Воспользоваться методом интегрирования подведением под знак дифференциала или замены переменных, чтобы получить табличный интеграл
    \item [4.2] ???
    \item [4.3] PROFIT
    \end{enumerate}
\end{enumerate}

\newpage

\section{Вопрос №3}  %%%%%%%%%%%%%%%%% ВОПРОС 3
\begin{framed}Дать определение интеграла Римана. Доказать его свойства: линейность и интеграл от константы.
\end{framed}

Пусть функция $f(x)$ определена на отрезке $[a,b]$. Конечное множество точек $a = x_0 < x_1 < \ldots < x_n = b$ называют разбиением отрезка $[a,b]$ и обозначают $P = (x_0, x_1, \ldots, x_n)$. Для каждого $i = 1, \ldots, n$ отрезок $[x_{i-1}, x_i] \subseteq [a,b]$ назовем частичным отрезком разбиения $P$ и обозначим через $\Delta_i$. Его длину обозначим через $\Delta x_i = x_i - x_{i-1}~(i = 1, \ldots, n)$. Число $\lambda = max_{i = 1, \ldots, n}\Delta x_i $ называют параметром разбиения (или диаметром разбиения) $P$. Заметим, что $\lambda \geqslant (b-a)/n$.\\

Пусть на каждом частичном отрезке произвольным образом выбрана точка $\xi_i \in [x_{i-1}, x_i]$. Пару $(P, \xi)$, где $\xi = (\xi_1, \ldots, \xi_n)$, называют разбиением с отмеченными точками.\\

Сумму 
\[ \sigma(f;P, \xi) = \sum\limits_{i=1}^{n}f(\xi_i)\Delta x_i\]

называют интегральной суммой для функции $f(x)$, соответствующей разбиению $P = (x_0, \ldots, x_n)$ с отмеченными точками $\xi = (\xi_1, \ldots, \xi_n)$ отрезка $[a,b]$.\\

В множестве $\mathcal{P}$ разбиений с отмеченными точками данного отрезка $[a,b]$ рассмотрим следующую базу $\{b_d\}, d> 0:$
\[ b_d = \{ (P, \xi) \in \mathcal{P} : \lambda < d\}.\]

Проверим, что $\{b_d\}, d> 0$ --- действительно база в $\mathcal{P}$:
\begin{enumerate}
\item [1)] Для любого $d > 0$ элемент $b_d$ не пуст, так как существует разбиение отрезка $[a,b]$ на $n = [(b-a)/d] +1$ отрезков одинаковой длины $(b-a)/n <d$. В качестве отмеченных точек $\xi_i$ всегда можно взять середины отрезков. Поэтому $b_d \neq \varnothing$.
\item [2)] Если $d_1 >0$, $d_2 > 0$ и $d = min(d_1,d_2)$, то $b_{d/2} \subset b_d = b_{d_1} \cap b_{d_2}$ и $b_{d/2} \neq b_d$.
\end{enumerate}

Будем обозначать эту базу через $\lambda \to 0$. Предел по этой базе значений интегральных сумм для функции $f$, отвечающих разбиению с отмеченными точками отрезка $[a,b]$, называют \textbf{интегралом Римана (определенным интегралом)} от функции $f$ на отрезке $[a,b]$ и обозначают
\[ \int_{a}^b f(x) dx = \lim_{\lambda \to 0}\sum_{i=1}^n f(\xi_i) \Delta x_i.\]

\begin{theorem}[Линейность интеграла Римана] Пусть функции $f_1$и $f_2$ интегрируемы на отрезке $[a,b]$. Тогда при $\alpha_1, \alpha_2 \in \mathbb{R}$ функция $\alpha_1 f_1 + \alpha_2 f_2 $ также интегрируема на отрезке $[a,b]$ и 
\begin{equation}
\int_a^b (\alpha_1 f_1 + \alpha_2 f_2) (x) ~dx = \alpha_1 \int_a^b f_1(x) ~dx + \alpha_2 \int_a^b f_2(x)~dx. \label{eq:3:1}
\end{equation}
\end{theorem}
\begin{proof} Пусть $a < b$. Для некоторого разбиения $P = (x_0, \ldots, x_n)$ с отмеченными точками $\xi = (\xi_1, \ldots, \xi_n)$ отрезка $[a,b]$ имеем равенство
\begin{equation}
\sum_{i=1}^n(\alpha_1f_1(\xi_i) + \alpha_2 f_2 (\xi_i))\Delta x_i = \alpha_1 \sum_{i=1}^n f_1 (\xi_i) \Delta x_i + \alpha_2 \sum_{i=1}^n f_2 (\xi_i) \Delta x_i.\label{eq:3:2}
\end{equation}

В силу интегрируемости функций $f_1(x)$ и $f_2(x)$ на отрезке $[a,b]$, стоящие в правой части этого равенства интегральные суммы имеют пределы по базе $\lambda \to 0$. Но тогда существует конечный предел по этой базе и для интегральной суммы в левой части этого равенства, что в силу определения доказывает интегрируемость функции $\alpha_1f_1(x) + \alpha_2 f_2(x)$ на отрезке $[a,b]$. Переходя в обеих частях равенства \eqref{eq:3:2} к пределу, получаем равенсто из утверждения теоремы.\\

При $a = b$ все три интеграла в \eqref{eq:3:1} равны нулю. А при $a > b$, меняя местами $a$ и $b$, меняем знаки всех трех интегралов, а значит, этот случай сводится к случаю $a < b$.
\end{proof}

\begin{theorem}[Интеграл от константы]
\[ \int_a^b c~dx = c(b-a).\]
\end{theorem}
\begin{proof}
Интегральная сумма для константы не зависит от выбора разбиения $P$ и отмеченных точек $\xi$ и равна
\[ \sigma(f;P;\xi) = \sum_{i=1}^n c\Delta x_i = c(b-a).\]

Из теоремы о пределе постоянной функции получаем указанную формулу.
\end{proof}
\section{Вопрос №4} %%%%%%%%%%%%%%%%% ВОПРОС 4
\begin{framed}
Необходимое условие интегрируемости и его доказательство
\end{framed}
\begin{theorem}[необходимое условие интегрируемости]
\[ f \subset \mathcal{R}[a,b] \implies f \text{ ограничена на отрезке } [a,b].\]
\end{theorem}

\begin{proof}
Предположим противное: функция $f(x)$ интегрируема на отрезке $[a,b]$ и не ограничена на этом отрезке. Согласно определению интегрируемости, существует конечный предел интегральных сумм для этой функции на данном отрезке. Из теоремы о локальной ограниченности функции, имеющей предел, существует элемент базы $b_d$, на котором интегральная сумма ограничена, т.е.
\begin{equation}
\exists c > 0 ~~~ \left( \lambda(P) < d \implies |\sigma(f;P;\xi)| < c\right).\label{eq:4:1}
\end{equation}

Выберем одно из таких разбиений $(P, \xi)$  с отмеченными точками. Поскольку функция $f(x)$ не ограничена на отрезке $[a, b]$,
 то найдется частичный отрезок $\Delta_j$, на котором функция $f(x)$ является неограниченной. Разобьем интегральную сумму на два слагаемых и используем неравенство треугольника:
\begin{equation}
\begin{aligned} 
|\sigma(f;P;\xi)| &=  \left| f(\xi_j) \Delta_j + \sum_{i\neq j} f(\xi_i) \Delta x_i \right| \geqslant\\
&\geqslantя |f (\xi_j) \Delta x_j| - \left|  \sum_{i\neq j} f(\xi_i) \Delta x_i \right| = |f(\xi_j)| \Delta x_j - c_1,
\end{aligned} \label{eq:4:2}
\end{equation}

где $c_1$ --- значение второго слагаемого. Меняя только точку $\xi_j \in \Delta_j$, но не меняя остальные элементы разбиения $(P, \xi)$, слагаемое $f(\xi_j) \Delta x_j$ в интегральной сумме можно сделать сколь угодно большим по абсолютной величине, но это противоречит неравенствам \eqref{eq:4:1} и \eqref{eq:4:2}. Возникшее противоречие опровергает принятое предположение и доказывает утверждение теоремы.
\end{proof}

\section{Вопрос №5} %%%%%%%%%%%%%%%%% ВОПРОС 5
\begin{framed}
Критерий интегрируемости, его доказательство.
\end{framed}

\begin{theorem}[Критерий интегрируемости]
В введенных обозначениях
\[ f \in \mathcal{R}[a,b] \iff \lim_{\lambda \to 0} \sum_{i=1}^n \omega(f; \Delta_i) \Delta x_i = 0. \]
\end{theorem}
Докажем сначала два вспомогательных утверждения. Пусть разбиение $\tilde{P}$ отрезка $[a,b]$ получено из разбиения $P$ только добавлением некоторых новых точек. Будем называть разбиение $\tilde{P}$ \textbf{продолжением} разбиения $P$.\\

\begin{lemma}
Пусть $\tilde{P}$ --- продолжение разбиения $P$. $\xi, \tilde{\xi}$ --- отмеченные точки разбиений $P, \tilde{P}$ соответственно. Тогда
\begin{equation}
\left|\sigma(f; \tilde{P}, \tilde{\xi}) - \sigma(f; P, \xi)\right| \leqslant \sum_{i=1}^n \omega (f; \Delta_i) \Delta x_i. \label{eq:5:1}
\end{equation}
\end{lemma}
\begin{proof}
Введем следующие обозначения. При построении продолжения $\tilde{P}$ некоторые (может и не все) частичные отрезки $\Delta_i = [x_{i-1}, x_i]$ разбиения $P$ сами подвергаются разбиению $x_{i-1} = x_{i0} < x_{i1} < \ldots < x_{ik_i} = x_i$. Поэтому будем нумеровать точки разбиения $\tilde{P}$  двумя индексами. В записи $x_{ij}$ первый индекс означает, что $x_{ij}  \in \Delta_i$, а второй индекс есть порядковый номер точки на отрезке $\Delta_i$. Положим также $\Delta x_{ij} = x_{ij} - x_{ij - 1}$ и $\Delta_{ij} = [x_{ij - 1}, x_{ij}]$.\\

Пусть $\xi = (\xi_1, \ldots, \xi_n)$ и $\tilde{\xi} = (\xi_{11}, \ldots, \xi_{nk_n})$  --- отмеченные точки разбиений $P$ и $\tilde{P}$ соответственно. Нумерация точек $\tilde{\xi}$ выбрана так, что $\xi_{ij} \in \Delta_{ij}$. Используя определения, неравенство треугольника и равенства $\Delta x_i = \Delta x_{i1} + \ldots + \Delta x_{ik_i}$ при $i = 1, \ldots, n$, получаем
\[ | \sigma(f; \tilde{P}, \tilde{\xi}) - \sigma(f; P; \xi) | = \left| \sum_{i=1}^n \sum_{j=1}^{k_i} f(\xi_{ij}) \Delta x_{ij} - \sum_{i=1}^n f(\xi_i) \Delta x_{i}  \right| =\]
\[= \left| \sum_{i=1}^n \sum_{j=1}^{k_i}f(\xi_{ij}) \Delta x_{ij} - \sum_{i=1}^n \sum_{j=1}^{k_i} f(\xi_i) \Delta x_{ij} \right| = \left| \sum_{i=1}^n \sum_{j=1}^{k_i} (f(\xi_{ij}) - f(\xi_i)) \Delta x_{ij} \right| \leqslant \]
\[ \leqslant \sum_{i=1}^n \sum_{j=1}^{k_i} \left| f(\xi_{ij}) - f(\xi_i) \right| \Delta x_{ij} \leqslant \sum_{i=1}^n \sum_{j=1}^{k_i} \omega(f; \Delta_i) \Delta x_{ij} = \sum_{i=1}^n \omega (f; \Delta_i) \Delta x_i. \]
\end{proof}

Пусть $f \in \mathcal{R}[a,b]$, а $P$ --- произвольное разбиение отрезка $[a,b]$. Тогда из необходимого условия интегрируемости следует ограниченность функции $f$ на отрезке $[a,b]$, и поэтому определены значения:
\[ M_i = \sup_{x \in \Delta_i} f(x),~~~ m_i = \inf_{x \in \Delta_i} f(x),~~~ S(f;P) = \sum_{i = 1}^n M_i \Delta x_i, ~~~ s(f; P) = \sum_{i = 1}^n m_i \Delta x_i.\]

Суммы $S(f;P)$ и $s(f; P)$ называют \textbf{верхней} и \textbf{нижней суммами Дарбу} для функции $f(x)$, соответствующими фиксированному разбиению $P$. Из определений имеем
\begin{equation}
s(f;P) \leqslant S(f;P). \label{eq:5:1.9}
\end{equation}

\begin{lemma}
Пусть $f \in \mathcal{R}[a,b]$, а $P$ --- какое-либо разбиение отрезка $[a,b]$. Тогда для любого $\varepsilon > 0$ найдется такой набор $\xi$ отмеченных точек, что
\begin{equation}
S(f;P) < \sigma (f; P, \xi) + \varepsilon. \label{eq:5:2}
\end{equation}
\end{lemma}
\begin{proof}
По определению чисел $M_i$ для каждого $i$ найдется точка $\xi_i \in \Delta _i$, в которой $M_i < f(\xi_i) + \dfrac{\varepsilon}{b - a}$. Тогда для $\xi = (\xi_1, \ldots, \xi_n)$ имеем
\[ S(f;P) = \sum_{i=1}^n M_i \Delta x_i < \sum_{i=1}^n \left( f(\xi_i) + \frac{\varepsilon}{b - a}\right) \Delta x_i = \sum_{i = 1}^n f(\xi_i) \Delta x_i + \varepsilon = \sigma(f; P, \xi) + \varepsilon. \]
\end{proof}

Аналогично доказывается, что для любого $\varepsilon > 0$ найдется такой набор $\xi'$ отмеченных точек, что
\begin{equation}
\sigma(f;P, \xi') - \varepsilon < s(f;P). \label{eq:5:3}
\end{equation}

\begin{proof}[Доказательство теоремы] Основано на критерии Коши существования предела по базе.\\

$\Leftarrow$: Пусть
\[ \lim_{\lambda \to 0} \sum_{i=1}^n \omega(f; \Delta_i) \Delta x_i = 0.\]

По определению предела по базе для любого $\varepsilon > 0$ найдется число $\delta > 0$ такое, что для любого разбиения $(P, \xi)$ с отмеченными точками отрезка $[a,b]$, параметр которого $\lambda < \delta$, имеет место соотношение
\begin{equation}
\left| \sum_{i=1}^n \omega(f; \Delta_i) \Delta x_i\right| < \frac{\varepsilon}{2}. \label{eq:5:4}
\end{equation}

Если теперь $(P', \xi')$ и $(P'', \xi'')$  --- разбиения с отмеченными точками отрезка $[a,b]$, параметры которых удовлетворяют условиям $\lambda' < \delta,~ \lambda'' < \delta$, то рассмотрим разбиение $\tilde{P} = P' \cup P''$, являющееся продолжением обоих разбиений $P', P''$. Из неравенств \eqref{eq:5:1} и \eqref{eq:5:4} следует, что
\[ \left|\sigma(f; \tilde{P}, \tilde{\xi}) - \sigma(f; P', \xi') \right| < \frac{\varepsilon}{2}, ~~~~ \left|\sigma(f; \tilde{P}, \tilde{\xi}) - \sigma(f; P'', \xi'')\right| < \frac{\varepsilon}{2}. \]

А значит, из неравенства треугольников получаем
\[ \left| \sigma(f; P', \xi') - \sigma(f; P'', \xi'')\right| < \varepsilon.\]

В силу критерия Коши существует предел $\lim\limits_{\lambda \to 0} \sigma(f; P, \xi)$, т.е. $f \in \mathcal{R}[a,b]$.\\

$\Rightarrow$: Если  $f \in \mathcal{R}[a,b]$, то существует предел  $\lim\limits_{\lambda \to 0} \sigma(f; P, \xi)$. В силу критерия Коши для любого $\varepsilon > 0$ найдется такое число $\delta > 0$, что для разбиений $(P, \xi')$ и $(P, \xi'')$ с отмеченными точками отрезка $[a,b]$, параметр которых $\lambda < \delta$, имеет место соотношение
\begin{equation}
\left| \sigma(f;P, \xi) - \sigma(f;P, \xi')\right| < \varepsilon. \label{eq:5:5}
\end{equation}


Из соотношений \eqref{eq:5:1.9} - \eqref{eq:5:3} и \eqref{eq:5:5} следует, что
\[ \left| S(f;P) - s(f;P)\right| \leqslant | \sigma(f;P, \xi) - \sigma(f; P, \xi') + 2 \varepsilon | < 3 \varepsilon,\]

а значит
\[ \lim_{\lambda \to 0} \left( S(f;P) - s(f;P)\right) = 0.\]

Но $\omega(f;\Delta_i) = M_i - m_i$, поэтому
\[ \lim_{\lambda \to 0} \sum_{i = 1}^n \omega(f; \Delta_i) \Delta x_i =  \lim_{\lambda \to 0} \sum_{i = 1}^n (M_i - m_i) \Delta x_i =\lim_{\lambda \to 0} \left( S(f;P) - s(f;P)\right) = 0.\]
\end{proof}

\section{Вопрос №6} %%%%%%%%%%%%%%%%% Вопрос 6
\begin{framed}
Доказательство интегрируемости непрерывных функций и функций с конечным числом
точек разрыва.
\end{framed}
\begin{remark}Если функция $f(x)$ непрерывна на отрезке $[a,b]$, то она интегрируема на этом отрезке.
\end{remark}
\begin{proof}
По теореме Кантора, поскольку функция $f(x)$ непрерывна на отрезке $[a,b]$, она равномерно непрерывна на нем. Поэтому для любого $\varepsilon > 0$ найдется такое $\delta = \delta(\varepsilon) > 0$, что отрезок $[a,b]$ можно разбить на частичные отрезки длиной меньше $\delta$, на каждом из которых колебание функции $f(x)$ будет меньше $\dfrac{\varepsilon}{b - a}$, т.е. при диаметре $\lambda(P) < \delta$ разбиения $P$ будет выполнено неравенство $0 \leqslant \omega(f; \Delta_i) < \dfrac{\varepsilon}{b-a}$. Умножая это неравенство на длину $\Delta x_i > 0$  частичного отрезка и суммируя по $i$, получаем
\[ 0 \leqslant \sum_{i = 1}^n \omega(f; \Delta_i) \Delta x_i < \sum_{i = 1}^n \frac{\varepsilon}{b-a} \Delta x_i = \varepsilon,\]

что в силу критерия интегрируемости доказывает утверждение теоремы.
\end{proof}

\begin{remark} Ограниченная с конечным числом точек разрыва на отрезке функция интегрируема на этом отрезке.
\end{remark}
\begin{proof}
Пусть функция $f$ ограничена и имеет $k$ точек разрыва на отрезке $[a,b]$. Проверим выполнимость условия критерия интегрируемости.\\

Так как функция $f$ ограничена, то определено и конечно колебание $\omega(f; [a,b])$ этой функции на отрезке $[a,b]$. Обозначим его через $\omega$. При заданном $\varepsilon > 0$ посторим $\delta_1$-окрестности каждой из $k$ точек разрыва функции $f$ на $[a,b]$ (значение $\delta_1$ выберем позже). Дополнительное к объединению этих окрестностей множество точек отрезка $[a,b]$ состоит из конечного числа отрезков, на каждом из которых $f$ непрерывна и, значит, равномерно непрерывна. Поскольку таких отрезков конечное число, по $\varepsilon > 0$ можно указать $\delta_2 = \delta_2(\varepsilon) > 0$ так, что на любом отрезке $\Delta$, длина которого меньше $\delta_2$  и который полностью содержится в одном из указанных выше отрезков непрерывности $f$, будем иметь $\omega(f; \Delta) < \dfrac{\varepsilon}{2(b-a)}$.\\

Возьмем теперь число $\delta = \min\{\delta_1, \delta_2\}$. Пусть $P$ --- произвольное разбиение отрезка $[a,b]$, для которого $\lambda(P) < \delta$. Сумму из критерия интегрируемости, отвечающую разбиению $P$, разобьем на две части:
\[ \sum_{i = 1}^n \omega(f; \Delta_i) \Delta x_i = \sum_{i'} \omega(f; \Delta_i) \Delta x_i + \sum_{i \neq i'} \omega(f; \Delta_i) \Delta x_i.\]

В сумму $\sum\limits_{i'}$ включим те слагаемые, которые отвечают отрезкам $\Delta_i$ разбиения $P$, не имеющим общих точек с постороенными $\delta_1$-окрестностями точек разрыва. Для таких отрезков $\Delta_i$ имеем $\omega(f; \Delta_i) < \dfrac{\varepsilon}{2(b-a)}$, поэтому
\[ \sum_{i'} \omega(f; \Delta_i) \Delta x_i < \frac{\varepsilon}{2(b-a)} \sum_{i'} \Delta x_i < \frac{\varepsilon}{2(b-a)} (b-a) = \frac{\varepsilon}{2}.\]

Сумма длин оставшихся отрезков разбиения $P$ меньше
\[ (\delta + 2 \delta_1 + \delta)k \leqslant 4 \delta_1 k = \frac{\varepsilon}{2\omega},\]

если положить $\delta_1 = \dfrac{\varepsilon}{8\omega k}$. Поэтому
\[ \sum_{i \neq i'} \omega(f; \Delta_i) \Delta x_i \leqslant \omega \sum_{i \neq i'} \Delta x_i < \omega \frac{\varepsilon}{2\omega} = \frac{\varepsilon}{2}.\]

Таким образом, мы получаем, что при $\lambda(P) < \delta$
\[ \sum_{i = 1}^n \omega(f; \Delta_i) \Delta x_i < \varepsilon, \]

т.е. выполнено условие критерия интегрируемости поэтому $f \in \mathcal{R}[a,b]$.
\end{proof}

\newpage
\section{Вопрос №7} %%%%%%%%%%%%%%%%%%%%% Вопрос 7

\begin{framed}
Сформулировать и доказать свойство аддитивности определенного интеграла.
\end{framed}

\begin{theorem} \label{th:7:1}
Если функция интегрируема на отрезке $[a,b]$, то она интегрируема и на любом меньшем отрезке $[c,d] \subset [a,b]$.
\end{theorem}
\begin{proof}
Пусть $P$ --- разбиение отрезка $[c,d]$. Добавив к $P$ некоторые точки, достроим $P$ до разбиения $P'$ отрезка $[a,b]$, но так, чтобы диаметр $\lambda'$ разбиения $P'$ был не больше диаметра $\lambda$ разбиения $P$. Тогда
\begin{equation}
0 \leqslant \sum_{P} \omega(f |_{[c,d]}; \Delta_i) \Delta x_i \leqslant \sum_{P'} \omega(f; \Delta_i) \Delta x_i, \label{eq:7:1}
\end{equation}

где $\sum\limits_{P}$ --- сумма по всем отрезкам разбиения $P$, а $\sum\limits_{P'}$ --- сумма по всем отрезкам разбиения $P'$.\\

Так как $\lambda \geqslant \lambda' \geq 0$, то $\lambda' \to 0$ при $\lambda \to 0$. Переходя в неравенстве \eqref{eq:7:1} к пределу по базе $\lambda \to 0$ и используя теорему о пределе промежуточной функции и критерий интегрируемости, получаем $f \in \mathcal{R} [a,b] \implies f \in \mathcal{R} [c,d].$
\end{proof}

Определенный интеграл Римана обобщается на случай $a \geqslant b$, при этом перестановка пределов интегрирования в определенном интеграле изменяет его знак, а интегралу с $a = b$ приписывают нулевое значение:
\begin{equation}
\int_a^b f(x)~dx = - \int_b^a f(x)~dx \text{ при } a > b, ~~~~~ \int_a^a f(x)~dx = 0.\label{eq:7:2}
\end{equation}

\begin{theorem}
Если функция $f(x)$ интегрируема на наибольшем из отрезков $[a,b]$, $[a,c]$, $[c,b]$, то она интегрируема на двух других отрезках и справедливо равенство
\begin{equation}
\int_a^b f(x)~dx = \int_a^c f(x) ~dx + \int_c^b f(x)~dx, \label{eq:7:3}
\end{equation}

каково бы ни было взаимное расположение точек $a$, $b$ и $c$.
\end{theorem}
\begin{proof}
Предположим сначала, что $a < c < b$ и функция $f(x)$  интегрируема на отрезке $[a,b]$. На основании теоремы \ref{th:7:1} заключаем, что $f(x)$ интегрируема на отрезках $[a,c]$ и $[c,b]$. Поэтому в дальнейших рассуждениях можно использовать некоторый специальный вид разбиений этих отрекзов.

\begin{center}
\begin{tikzpicture}
\draw[thick, ->] (-3.5, 0) -- (3,0) node[below]{$x$};
\draw (2.49, 0) -- (2.49, .25) node at (2.49, -.2) {$b$};
\draw (-2.98, 0) -- (-2.98, .25) node at (-2.98, -.2) {$a$};
\draw (-1, 0) -- (-1, .25) node at (-1, -.2) {$c$};
\draw (-1, 0) arc (70:110:2.9cm) node at (1, .6) {$T_m$};
\draw (2.5, 0) arc (-70:-110:8cm) node at (-.2, -1) {$T_n$};
\draw (2.5, 0) arc (70:110:5.1cm) node at (-2, .6) {$T_k$};
\draw node at (0, -2) {Рис. 1};
\end{tikzpicture} \label{pic:7:1}
\end{center}

Рассмотрим разбиение $T_n$ отрезка $[a,b]$ на $n$ частичных отрезков, причем точку $c$ будем считать одной из точек деления этого отрезка (\hyperref[pic:7:1]{рис. 1}). Пусть при этом $T_k$ и $T_m$ --- разбиения отрезков $[a,c]$ и $[c,b]$ на $k$ и $m$ частичных отрезков соответственно $(k + m = n)$. Тогда интегральную сумму функции $f(x)$ для разбиения $T_n = T_k \cup T_m$ отрезка $[a,b]$ можно записать в виде
\begin{equation}
\sum_{i=1}^n f(\xi_i) \Delta x_i = \sum_{i=1}^k f(\xi'_i) \Delta x'_i + \sum_{i=1}^m f(\xi''_i) \Delta x''_i, \label{eq:7:4}
\end{equation}

где первое слагаемое справа соответствует разбиению $T_k$ отрезка $[a,c]$, а второе --- разбиению $T_m$ отрезка $[c,b]$. Параметры $\lambda_n, \lambda_k, \lambda_m$ разбиений $T_n, T_k, T_m$ связаны неравенствами $\lambda_k \leqslant \lambda_n$, $\lambda_m \leqslant \lambda_n$. Поэтому при $\lambda_n \to 0$ имеем также $\lambda_k \to 0$ и $\lambda_m \to 0$. А значит, переходя в \eqref{eq:7:4} к пределу по базе $\lambda_n \to 0$, получаем равенство \eqref{eq:7:3}.\\

Пусть теперь $a < b < c$. Тогда на основании доказанного имеем
\[ \int_a^c f(x)~dx = \int_a^b f(x)~dx + \int_b^c f(x)~dx.\]

Разрешая это равенство относительно интеграла по отрезку $[a,b]$ и используя \eqref{eq:7:2}, получаем
\[ \int_a^b f(x)~dx = \int_a^c f(x)~dx - \int_b^c f(x)~dx = \int_a^c f(x)~dx + \int_c^b f(x)~dx.\]

Аналогично это свойство можно доказать для любого другого взаимного расположения точек $a$, $b$ и $c$.
\end{proof}

\section{Вопрос №8} %%%%%%%%%%%%%% Вопрос 8
\begin{framed}
Доказать свойства определенного интеграла: монотонность, теорему о сохранении интегралом знака подынтегральной функции, теорему об оценке модуля и теорему об оценке определенного интеграла.
\end{framed}

\begin{theorem} \label{th:8:1}
~\\ % Why do I have to do this?
\begin{enumerate}
    \item[a)] Если $a < b$, функция $f$ интегрируема на отрезке $[a,b]$ и $f(x) \geqslant 0~ \forall x \in [a,b]$, то
    \[ \int_a^b f(x)~dx \geqslant 0.\]
    \item[b)] Если, кроме того, существует точка $x' \in [a,b]$, в которой $f(x)$ непрерывна и $f(x') > 0$, то
    \[ \int_a^b f(x)~dx > 0.\]
\end{enumerate}
\end{theorem}
\begin{proof}
Так как $f(x) \geqslant 0 ~ \forall x \in [a,b]$ и $a < b$, то для любого разбиения $P$ с отмеченными точками $\xi$ отрезка $[a,b]$ имеем $\Delta x_i > 0$ и
\[ \sum_{i=1}^n f(\xi_i) \Delta x_i \geqslant 0. \]

Переходя в этом неравенстве к пределу по базе $\lambda \to 0$ получаем утверждение a).\\

Согласно свойствам функции, непрерывной в точке, и условию теоремы, существует окрестность точки $x'$ (или полуокрестность этой точки, если $x'$ совпадает с одним из концов отрезка), в которой $f(x) \geqslant \frac{1}{2} f(x') = A > 0$. Выделим в этой окрестности (или полуокрестности) отрезок $[\alpha, \beta]$. Тогда в силу аддитивности определенного интеграла, утверждения a) и свойства линейности имеем
\[ 
\begin{aligned}
\int_a^b f(x)~dx &= \int_a^{\alpha} f(x)~dx + \int_{\alpha}^{\beta} f(x)~dx + \int_{\beta}^b f(x)~dx \geqslant\\
&\geqslant \int_{\alpha}^{\beta} f(x)~dx = \int_{\alpha}^{\beta} (f(x) - A)~dx + \int_{\alpha}^{\beta} A~dx \geqslant A(\beta - \alpha) > 0.
\end{aligned}
\]
\end{proof}
\begin{theorem}[Монотонность интеграла] \label{th:8:2}
Если $a < b$, функции $f_1$ и $f_2$ интегрируемы на отрезке $[a,b]$ и $f_1(x) \geqslant f_2(x) ~\forall x \in [a,b]$, то 
\[ \int_a^{b} f_1(x)~dx \geqslant \int_a^b f_2(x)~dx.\]
\end{theorem}
\begin{proof}s
В силу свойства линейности определенного интеграла функция $f_1(x) - f_2(x)$ интегрируема на отрезке $[a,b]$ и, применяя теорему \ref{th:8:1}, получаем
\[ \int_a^b f_1(x)~dx - \int_a^b f_2 (x)~dx = \int_a^b (f_1(x) - f_2(x))~dx \geqslant 0,\]

так как по условию $f_1(x) - f_2(x) \geqslant 0~\forall x \in [a,b]$.
\end{proof}

\begin{theorem}[Теорема об оценке]
Если $a < b$, функция $f(x)$ интегрируема на отрезке $[a,b]$ и $m \leqslant f(x) \leqslant M~\forall x \in (a,b)$, то
\[ m(b-a) \leqslant \int_a^b f(x)~dx \leqslant M(b-a).\]
\end{theorem}
\begin{proof}
Применим теорему \ref{th:8:2} к неравенству $m \leqslant f(x) \leqslant M$ на отрезке $[a,b]$.
\end{proof}
\begin{theorem}[Теорема об оценке модуля интеграла]
Если функция $f(x)$ интегрируема на отрезке $[a,b]$, то ее модуль $|f(x)|$ есть также интегрируемая функция на отрезке $[a,b]$ и
\begin{equation}
\left| \int_a^b f(x)~dx\right| \leqslant \left| \int_a^b |f(x)|~dx\right|. \label{eq:8:1}
\end{equation}
\end{theorem}

\begin{proof}
Учитывая неравенство $\left| |f(\xi)| - |f(\eta)| \right| \leqslant \left| f(\xi) - f(\eta) \right|$, верное для любых точек $\xi, \eta \in [a,b]$, заключаем, что $\omega(|f|; E) \leqslant \omega(f; E)$. Поэтому для произвольного разбиения отрезка $[a,b]$ имеем
\[ 0 \leqslant \sum_{i=1}^n \omega(|f|; \Delta_i) \Delta x_i \leqslant \sum_{i=1}^n \omega(f; \Delta_i) \Delta x_i.\]

Согласно критерию интегрируемости, предел по базе $\lambda \to 0$ правой части этого неравенства равен нулю. Следовательно, предел по этой базе центральной части также равен нулю, что по тому же критерию означает интегрируемость функции $|f(x)|$ на отрезке $[a,b]$.\\

Пусть $a < b$. Так как $-|f(x)| \leqslant f(x) \leqslant |f(x)|$ при $x \in [a,b]$, то по теореме \ref{th:8:2}:
\[ - \int_a^b |f(x)|~dx \leqslant \int_a^b f(x)~dx \leqslant \int_a^b |f(x)|~dx,\]
это равносильно неравенству \eqref{eq:8:1}.\\

При $a = b$ оба интеграла в \eqref{eq:8:1} равны нулю. А при $a > b$, меняя местами $a$ и $b$, меняем знаки обоих интегралов, а значит, этот случай сводится к случаю $a < b$.
\end{proof}

\section{Вопрос №9} %%%%%%%%%%%%%%%%%%%%%%% Вопрос 9
\begin{framed}
Доказать две теоремы о среднем для определенного интеграла.
\end{framed}
\begin{theorem}[Теорема о среднем значении]
Если функция $f(x)$  непрерывна на отрезке $[a,b]$, то на этом отрезке найдется хотя бы одна точка $c$, для которой справедливо равенство
\begin{equation}
\int_a^b f(x)~ dx = f(c) (b-a).\label{eq:9:1}
\end{equation}
\end{theorem}

\begin{proof}
Так как функция $f(x)$ непрерывна на отрезке $[a,b]$, то она интегрируема на нем. Кроме того, согласно теореме Вейерштрасса, непрерывная на отрезке функция достигает на этом отрезке своих наименьшего $m$ и наибольшего $M$ значений. Поскольку $m \leqslant f(x) \leqslant M~\forall x \in [a,b]$, то на основании теоремы об оценке можно записать 
\[ m \leqslant \frac{1}{n-a} \int_a^b f(x)~dx \leqslant M.\]

Обозначим среднюю часть этого неравенства через $\mu$. Тогда $\mu \in [m, M]$, причем функция $f(x)$ принимет значения $m$ и $M$.  Согласно теореме Больцано --- Коши о промежуточном значении непрерывной функции, найдется хотя бы одна точка $c \in [a,b]$, в которой $f(c) = \mu$. Учитывая определение числа $\mu$, получаем утверждение теоремы.
\end{proof}
\begin{theorem} [Обобщенная теорема о среднем значении]
Если на отрезке $[a,b]$ функция $f(x)$ непрерывна, а функция $g(x)$ интегрируема и знакопостоянна, то на этом отрезке найдется хотя бы одна точка $c$, для которой справделиво равенство
\begin{equation}
\int_a^b f(x) g(x)~dx = f(c) \int_a^b g(x)~dx. \label{eq:9:2}
\end{equation}
\end{theorem}
\begin{proof}
Пусть $a < b$ и $g(x) \geqslant 0~\forall x \in [a,b]$. Так как функция $f(x)$ непрерывна на отрезке $[a,b]$, то, согласно теореме Вейерштрасса, она достигает на этом отрезке своего наименьшего $m$ и наибольшего $M$ значений и при этом $m g(x) \leqslant f(x)g(x) \leqslant Mg(x)~\forall x \in [a,b]$. В силу свойств монотонности и линейности интеграла имеем
\begin{equation}
m \int_a^b g(x)~dx \leqslant \int_a^b f(x) g(x) ~dx \leqslant M \int_a^b g(x)~dx. \label{eq:9:3}
\end{equation}

Согласно теореме \ref{th:8:1}, интеграл от неотрицательной функции неотрицателен, т.е.
\[ I = \int_a^b g(x)~dx \geqslant 0.\]

Если $I = 0$, то интеграл в средней части \eqref{eq:9:3} также равен нулю и \eqref{eq:9:2} верно для любой точки $c \in [a,b]$. Если же $I > 0$, то, разделив \eqref{eq:9:3} на I, получим
\[ m \leqslant \frac{1}{I} \int_a^b f(x) g(x) ~dx \leqslant M.\]

Обозначим среднюю часть этого неравенства через $\mu$. Так как $\mu \in [m, M]$, то, согласно теореме Больцано --- Коши, найдется хотя бы одна точка $c \in [a,b]$, в которой  $f(c) = \mu$. Отсюда с учетом определения числа $\mu$ следует \eqref{eq:9:2}.\\

Аналогичное доказательство \eqref{eq:9:2} в случае $g(x) \leqslant 0~\forall x \in [a,b]$.\\

При $a = b$ оба интеграла в \eqref{eq:9:2} равны нулю. А при $a > b$, меняя местами $a$ и $b$ меням знаки обоих интегралов, а значит, этот случай сводится к случаю $a < b$.
\end{proof}

\section{Вопрос №10} %%%%%%%%%%%%%%%%%%%%%% Вопрос 10
\begin{framed}
Определенный интеграл с переменным верхним пределом. Доказать теорему о непрерывности интеграла c переменным верхним пределом.
\end{framed}
\begin{theorem}
Если функция $f(x)$ интегрируема на отрезке $[a,b]$, то функция
\begin{equation}
F(x) = \int_a^x f(t)~dt \label{eq:10:1}
\end{equation}
называемая \textbf{определенным интегралом с переменным верхним пределом}, определена и непрерывна на $[a,b]$.
\end{theorem}
\begin{proof}
Так как функция $f(x)$ интегрируема на отрезке $[a,b]$, то она интегрируема на любом отрезке $[a,x] \subseteq [a,b]$, а значит, функция $F(x)$ определена для любого $x \in [a,b]$.\\

В силу необходимого условия интегрируемости функция $f(x)$ ограничена на отрезке $[a,b]$, т.е. $|f(x)| \leqslant M$ при $x \in [a,b]$ для некоторого $M > 0$. Придадим произвольному $x_0 \in [a,b]$ приращение $\Delta x$, не выводящее точку $x_0 + \Delta x$ за пределы отрезка $[a,b]$. Тогда в силу аддитивности определенного интеграла приращение функции $F(x)$, соответствующее приращению $\Delta x$, можно представить в виде
\[ \Delta F = F(x_0 + \Delta x) - F(x_0) = \int_a^{x_0 + \Delta x} f(t)~dt - \int_a^{x_0} f(t)~dt = \int_{x_0}^{x_0 + \Delta x} f(t)~ dt.\]

Используя теорему об оценке модуля и монотонность интеграла, находим
\[ 0 \leqslant |\Delta F| = \left|\int_{x_0}^{x_0 + \Delta x} f(t)~ dt \right| \leqslant \left|\int_{x_0}^{x_0 + \Delta x} |f(t)|~ dt \right| \leqslant M|\Delta x|.\]

Устремляя $\Delta x$ к нулю, получаем $\lim\limits_{\Delta x \to 0} \Delta F = 0$, что и доказывает непрерывность функции $F(x)$ в точке $x_0$. При совпадении точки $x_0$ с одним из концов отрезка $[a,b]$ функция $F(x)$ будет непрерывна либо справа в точке $a$, либо слева в точке $b$. Так как $x_0$ является произвольной точкой отрезка $[a,b]$, то функция $F(x)$ непрерывна на этом отрезке.
\end{proof}
\end{document}  
